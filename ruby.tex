\documentclass{article}
\usepackage[german]{babel}
\usepackage[T1]{fontenc}
\usepackage[utf8]{inputenc}
\usepackage{listings}
\title{Seminar "Programmiersprachen" }
\author{Jarne Peters}
\date{07.01.2024} 
\begin{document}
\maketitle


\section*{Ruby}

Die objektorientierte Programmiersprache Ruby wurde 1995 vom Japaner Yukihiro Matsumoto entwickelt. \footnote{http://blog.nicksieger.com/articles/2006/10/20/rubyconf-history-of-ruby/Aufgerufen am 07.01.2024}Ruby ist besonders für seine einfache Syntax bekannt und zeichnet sich durch seine Dynamik und Interpretierbarkeit aus, was die schnelle und unkomplizierte Erstellung von Prototypen ermöglicht. Darüber hinaus bietet die objektorientierte Natur von Ruby eine elegante Möglichkeit, Daten und Funktionen in Form von Objekten zu organisieren.. Dies fördert eine klare Strukturierung des Codes und erleichtert die Wartung und Erweiterung der Anwendung. Die Nutzung von Ruby erstreckt sich über vielel Anwendungsbereichen, wobei die Webentwicklung im Vordergrund steht. Insbesondere das Ruby on Rails-Framework hat die Entwicklung von Webanwendungen revolutioniert. Ruby on Rails (oft als Rails abgekürzt) kann auf eine beeindruckende Erfolgsbilanz bei der Entwicklung Hunderttausender Anwendungen für Branchenführer wie GitHub, Shopify, Airbnb und Twitch zurückblicken.\footnote{https://rubyonrails.org/Aufgerufen am 07.01.2024} Rails bietet Entwicklern ein strukturiertes Framework, das Best Practices und Konventionen fördert und den Code effizienter und lesbarer macht. Die Popularität von Rails trug zur weiten Verbreitung von Ruby als Programmiersprache bei.Yukihiro Matsumoto, auch bekannt als Matz, hat Ruby entwickelt, um die besten Funktionen verschiedener Programmiersprachen, vor allem Perl, Smalltalk, Lisp und Python, zu kombinieren. Aber sein Hauptziel war die Entwicklung einer Sprache, mit der das Programmieren Spaß macht. In einem Vorwort zu seinem Buch 	\glqq Programming Ruby \grqq{} betonte Matsumoto: \glqq Rubys wesentliches Ziel ist ›Freude‹. Meines Wissens gibt es keine andere Sprache,die sich so sehr auf die Freude konzentriert. Rubys eigentliches Ziel ist es zu erfreuen – Sprachdesigner, Anwender,Sprachlerner, jeden.\grqq{}\footnote{(Matsumoto 2000, zit. nach der Seite
»Ruby (Programmiersprache)« 2018)/} Die schnelle Verbreitung von Ruby in Japan und später die Welt traten insbesondere nach der Veröffentlichung des Buches Programming Ruby auf, das die Grundlagen der Sprache erläuterte, und der Einführung des Web-Frameworks Ruby on Rails. Programmierern gefielen vor allem Rubys Einfachheit, Flexibilität und klare Syntax, die das Programmieren unterhaltsam und effektiv machten. Heute ist Ruby ein Open-Source-Projekt mit einer lebendigen und engagierten Community.\footnote{https://www.ruby-lang.org/de/community/ruby-core/Aufgerufen am 07.01.2024} Die Community trägt aktiv zur Weiterentwicklung der Sprache bei, aktualisiert sie regelmäßig und stellt verschiedene Bibliotheken, sogenannte „Gems“, auf RubyGems.org zur Verfügung. Diese Gems erweitern die Funktionalität von Ruby und stellen Entwicklern eine breite Palette von Tools für verschiedene Anwendungsfälle zur Verfügung. Insgesamt hat Ruby im Laufe der Jahre eine starke Position in der Entwicklergemeinschaft aufgebaut. Das liegt nicht nur an der technischen Leistungsfähigkeit, sondern auch an der Philosophie des Spaßes und der Benutzerfreundlichkeit, die Matsumoto in die Sprache integriert hat. Ruby ist dadurch zu einer einflussreiche Programmiersprache geworden, die Programmierer auf der ganzen Welt dazu bringt, Spaß am Programmieren zu haben.Dabei legte er vor allem wert auf das Prinzip "Principle of Least Surprise", wodurch alles so funktionieren sollte, wie es intuitiv zu erwartet ist. So ist es zum Beispiel egal ob man „and“ oder „\&\&“benutzt.\footnote{https://www.artima.com/articles/the-philosophy-of-ruby\#part4Aufgerufen am 07.01.2024}
Ruby ist vor allem wegen seiner einfachen Syntax beliebt. Der Code ist vergleichsweise einfach zu verstehen für jeden, der mit Programmiersprachen vertraut ist. Dabei kann man Ruby auf Linux, Mac, Unix und Windows ohne Probleme verwenden.\footnote{https://www.ruby-lang.org/en/documentation/faq/3/Aufgerufen am 07.01.2024}
Ruby strebt stets danach, die intuitivsten und vorhersehbarsten Konventionen beizubehalten. Diese Ausrichtung auf Benutzerfreundlichkeit und Klarheit in der Sprachgestaltung ermöglicht es Entwicklern, sich auf das Wesentliche zu konzentrieren und Spaß beim Programmieren zu haben. Insgesamt ist Ruby darauf ausgelegt, eine Balance zwischen objektorientierter Struktur, übersichtlicher Syntax und Benutzerfreundlichkeit zu bieten, um eine positive und produktive Entwicklungsatmosphäre zu schaffen.
In Ruby ist alles komplett objektorientiert, wodurch jeder Wert in Ruby ein Objekt ist und jede Funktion eine Methode. Ruby macht auch keine Ausnahmen für primitive Datentypen wie Integers, Booleans oder Null – sie existieren in dieser Form nicht. Man kann zum Beispiel die zahl 5 als Objekt benutzten indem man diese mit „5.“ anspricht.Zum Beispiel:
\begin{lstlisting}
5.times { print "Hello" }
\end{lstlisting}
Hier wird nun 5 mal Hello ausgegeben.
\footnote{https://www.ruby-lang.org/en/about/Aufgerufen am 07.01.2024}
Ruby ist aufgrund seiner Vielseitigkeit in verschiedenen Anwendungsbereichen weit verbreitet. Insbesondere hat es sich als Hauptakteur in der Webentwicklung etabliert, hauptsächlich durch das beliebte Ruby on Rails-Framework. Laut JetBrains finden 64\% aller Arten von Softwareanwendungen in Ruby Anwendung, wobei der Großteil für Websites (64\%), Hilfsprogramme (22\%), Systemsoftware (13\%) und Datenbanken (11\%) verwendet wird.
Interessanterweise bleibt Japan ein bedeutender Akteur in der Ruby-Nutzung, wobei 9\% aller Ruby-Anwendungen aus dem Land stammen. \footnote{https://www.jetbrains.com/de-de/lp/devecosystem-2021/ruby/Aufgerufen am 07.01.2024}
Bekannte Projekte, die Ruby nutzen, sind zum Beispiel Webanwendungen wie GitHub und Shopify. Darüber hinaus wird Ruby in anderen Projekten, wie der Entwicklung von Simulationen für Motorola oder dem Morpha-Projekt. In letzterem wurde Ruby verwendet, um die Reaktionskontrolle von Robotern für Siemens zu implementieren.\footnote{https://www.ruby-lang.org/en/documentation/success-stories/Aufgerufen am 07.01.2024}
Zur Installation von Ruby kann man entweder direkt auf der Website die gewünschte Version herunterladen oder auch einen version manager wie rbenv oder RVM benutzten.Mit diesen kann man leicht zwischen verschiedenen Versionen wechseln und sogar ganz einfach andere Implementationen von Ruby herunterladen.\footnote{https://www.ruby-lang.org/de/downloads/Aufgerufen am 07.01.2024}
\section*{Motivation zur Entwicklung von Ruby:}
Die Motivation Ruby zu entwickel bekam Matsumoto 1993. Dabei wollte er Sprache erschaffen die so einfach ist wie Lisp, Objektorientiert wie Smalltalk und mächtiger als Perl.\footnote{https://web.archive.org/web/20181027195101/http://blade.nagaokaut.ac.jp/cgi-bin/scat.rb/ruby/ruby-talk/179642Aufgerufen am 07.01.2024}
In einem post von 1999 in der Ruby-talk mailing list schrieb er dazu:\glqq Ich habe mit meinem Kollegen über die Möglichkeit einer objektorientierten Skriptsprache gesprochen. Ich kannte Perl (Perl4, nicht Perl5), mochte es aber nicht wirklich, weil es nach einer Spielzeug-Sprache roch (und immer noch riecht). Die objektorientierte Sprache schien sehr vielversprechend. Ich kannte damals Python. Aber ich mochte es nicht, weil ich nicht dachte, dass es eine echte objektorientierte Sprache war – die OO-Funktionen schienen wie ein Zusatz zur Sprache zu sein. Als Sprachenthusiast und OO-Fan seit 15 Jahren wollte ich wirklich eine echte objektorientierte, einfach zu verwendende Skriptsprache. Ich suchte danach, konnte aber keine finden. Also beschloss ich, sie selbst zu entwickeln.\grqq{}\footnote{https://ruby-doc.org/docs/ruby-doc-bundle/FAQ/FAQ.htmAufgerufen am 07.01.2024l}
Ruby sollte dabei zwischen Funktionaler und imperativer Programmierung ausbalanciert sein.\footnote{https://www.ruby-lang.org/de/about/Aufgerufen am 07.01.2024}
\section*{Geschichte Von Ruby:}
Am 21. Dezember 1995 erschien die erste Version von Ruby.\footnote{http://blog.nicksieger.com/articles/2006/10/20/rubyconf-history-of-ruby/\label{rubyh}Aufgerufen am 07.01.2024} Bereits in der ersten Version waren viele Features wie objektorientiertem Design, Klassenvererbung, Mixins, Iteratoren, Closures, Exception Handling und Garbage Collection bereits enthalten.\footref{rubyh} Im Jahr 1997 wurde Yukihiro Matsumoto von netlap.jp angestellt, um Vollzeit an der Weiterentwicklung von Ruby zu arbeiten.\footref{rubyh} In den darauf folgenden Jahren wurden mehrere stabile Versionen von Ruby veröffentlicht.
Im Jahr 1998 ist die erste englischen Ruby-Website durch Matsumoto erschienen, gefolgt von der Veröffentlichung des ersten Buches über Ruby in Japan im Jahr 1999, das er zusammen mit Keiju Ishitsuka verfasste. Bis zum Jahr 2000 hatte Ruby bereits an Popularität gewonnen und überholte sogar Python in Japan. Die Veröffentlichung des englischsprachigen Buches \grqq Programming Ruby\grqq{} verstärkte Rubys Aufstieg im Westen.\footref{rubyh}
Im Jahr 2003 erschien die Version 1.8, die lange Zeit als stabile Version diente, bis sie schließlich im Jahr 2013 eingestellt wurde.\footref{rubyh} Der Durchbruch von Ruby erfuhr einen weiteren Schub durch das Webframework Ruby on Rails, das 2004 eingeführt wurde und die Beliebtheit der Sprache steigerte.
Am 31. Oktober 2011 kam die Version 1.9 heraus, die bedeutende Veränderungen wie Blockvariablen und eine neue Socket-API (IPv6) mit sich brachte.\footnote{https://www.ruby-lang.org/en/news/2011/10/31/ruby-1-9-3-p0-is-released/Aufgerufen am 07.01.2024} Am 24. Februar 2013 folgte die Version 2.0, die mehrere neue Funktionen einführte, darunter Method Keyword Arguments, eine neue Literal-Syntax zur Erstellung von Symbolarrays und eine Konvertierung um Objekte in Hashes zu verwandeln. Zudem wurde der Garbage Collector mit der Version 2.2 aktualisiert, was eine inkrementelle Funktionsweise ermöglichte und auch die Sammlung von Symbolen ermöglichte. Des Weiteren wurde nun Unicode 7 unterstützt, und in den folgenden Updates lag der Fokus vor allem auf der Verbesserung der Performance.\footnote{https://www.ruby-lang.org/en/news/2013/02/24/ruby-2-0-0-p0-is-released/Aufgerufen am 07.01.2024}
Am 25. Dezember 2020 wurde schließlich die Version 3.0 veröffentlicht, auch als Ruby 3*3 bekannt, da sie dreimal schneller als die vorherige Version 2.0 seien sollte. Um diese Geschwindigkeitssteigerung zu erreichen, wurde ein Just-in-Time-Compiler eingeführt. Mit Version 3.1 kam dann ein neuer Just-in-Time-Compiler heraus, der von Shopify entwickelt wurde, um die Performance weiter zu optimieren und die Entwicklergemeinschaft mit noch leistungsfähigeren Tools auszustatten.\footnote{https://www.ruby-lang.org/en/news/2020/12/25/ruby-3-0-0-released/Aufgerufen am 07.01.2024}

\section*{Ruby on Rails:}

Ruby on Rails, auch einfach als Rails bekannt, ist ein Web Application Framework, das in der Programmiersprache Ruby entwickelt wurde. Im Jahr 2004 wurde Rails veröffentlicht und erlangte sehr schnell große Beliebtheit, da es zur richtigen Zeit eingeführt wurde und das erste Web Application Framework war, das zahlreiche neue Features einführte. Hinzu kam, dass es viele bewährte Designprinzipien und Konventionen effizient nutzen konnte.\footnote{https://rubyonrails.org/2005/12/13/rails-1-0-party-like-its-one-oh-ohAufgerufen am 07.01.2024}
Ein fundamentales Konzept, das in Ruby on Rails verwendet wird, ist das Model-View-Controller (MVC)-Muster.\footnote{https://guides.rubyonrails.org/getting\_started.htmll\label{rubyr}Aufgerufen am 07.01.2024} Dieses Muster organisiert die Struktur einer Anwendung in drei Hauptkomponenten: das Modell, den Controller und die Ansicht. Das Modell repräsentiert die Daten und ihre Beziehungen in der Datenbank. Der Controller handhabt Datenanfragen über URLs, während die Ansicht die Daten visuell darstellt. Diese klare Trennung der Verantwortlichkeiten erleichtert die Entwicklung und Wartung von Webanwendungen erheblich.
Rails folgt auch zwei wichtigen Prinzipien: "Don`t Repeat Yourself" (DRY) und "Convention over Configuration"\footref{rubyr}. Das DRY-Prinzip legt fest, dass Informationen nur einmal in der Datenbank hinterlegt werden müssen. Eine erneute Kopie im Quellcode ist daher nicht erforderlich. Dies fördert die Wiederverwendbarkeit von Code und verbessert die Wartbarkeit der Anwendung. Das Prinzip "Convention over Configuration" legt fest, dass das Framework sinnvolle Standardwerte bereitstellt, wodurch eine umfangreiche Konfiguration vermieden wird und Entwicklungszeit eingespart wird.\footref{rubyr}
Ein weiterer Grundsatz von Ruby on Rails ist, wie bei Ruby selbst auch, die Freude am Programmieren. Die Entwickler sollen Spaß dabei haben, Anwendungen zu erstellen, und das Framework ist darauf ausgerichtet, die Entwicklungsaufgaben so angenehm wie möglich zu gestalten. Dies schließt die Verwendung von APIs ein, die nach dem Prinzip "Bigger Smile" gestaltet sind, was bedeutet, dass die APIs so entwickelt werden sollen, dass die Entwickler Spaß beim Programmieren haben.\footnote{https://rubyonrails.org/doctrineAufgerufen am 07.01.2024} Dieser Fokus auf die Benutzerfreundlichkeit trägt dazu bei, dass Webanwendungen schnell und effizient realisiert werden können.
Nicht zuletzt haben viele bekannte Plattformen wie GitHub, Shopify und Twitch ihre Webanwendungen mit Ruby on Rails entwickelt. Insgesamt ist Ruby on Rails ein Vorreiter der modernen Web Application Frameworks und ist heutzutage immer noch ein solides und modernes Framework für Webanwendungen.\footnote{https://rubyonrails.org/Aufgerufen am 07.01.2024}

\section*{
Ruby Syntax :}

Die Syntax von Ruby ähnelt der von Perl oder Python. Klassen und Methoden werden durch Schlüsselwörter gekennzeichnet. Im Unterschied zu Perl benötigen Variablen in Ruby keinen Sigil als Präfix, wobei man diese auch verwenden kann, um den Geltungsbereich der Variablen festzulegen. Daher benötigt man kein \glqq self\grqq{}. vor jeder Instanzvariable.
Ein weiterer Unterschied zu Python oder Perl besteht darin, dass in Ruby alle Instanzvariablen vollständig privat für die Klassen sind und nur durch bestimmte Methoden wie attr\_writer oder attr\_reader offengelegt werden können.\footnote{https://en.wikipedia.org/wiki/Ruby\_syntaxAufgerufen am 07.01.2024}
Die Syntax von Ruby zeichnet sich durch ihre Einfachheit und Ausdrucksstärke aus. Ruby ist daher sehr intuitiv und einfach zu lesen. In einem Ruby-Programm werden Anweisungen einfach durch Zeilenumbrüche getrennt, und anders als viele Sprachen sind Semikolons am Zeilenende nicht notwendig.
Variablen werden durch ein vorangestelltes Dollarzeichen oder Kleinbuchstaben dargestellt, und die Typen werden dynamisch interpretiert.\footnote{https://www.ruby-lang.org/de/about/Aufgerufen am 07.01.2024} Datenstrukturen wie Arrays und Hashes bieten flexible Möglichkeiten zur Organisation von Informationen.
\section*{ Continuations:}
Ein interessantes Feature in Ruby sind Continuations. Mit der Continuation-Klasse ist es möglich, den aktuellen Zustand der Programmausführung zu erfassen und später wieder aufzunehmen. Die Verwendung von Continuations wird durch die Methode \glqq callcc\grqq{} ermöglicht, wenn man das Gem Continuations benutzt. Die callcc-Methode erzeugt eine Fortsetzung, die den aktuellen Zustand des Programms repräsentiert. Diese Fortsetzung kann dann an eine andere Stelle im Code übergeben werden. Wenn die Fortsetzung an späterer Stelle aufgerufen wird, wird die Programmausführung genau an dem Punkt wieder aufgenommen, an dem die Fortsetzung erstellt wurde.\footnote{https://ruby-doc.org/core-2.5.1/Continuation.htmlAufgerufen am 07.01.2024} Ein Beispiel verdeutlicht dies: \\ 
\begin{lstlisting}
require "continuation" 
arr = [ "Freddie", "Herbie", "Ron", "Max", "Ringo" ] 
callcc{|cc| $cc = cc} 
puts(message = arr.shift) 
$cc.call unless message =~ /Max/  
\end{lstlisting}
(Beispiel aus der Ruby-Dokumentation)\footnote{https://ruby-doc.org/core-2.5.1/Continuation.htmlAufgerufen am 07.01.2024} \\ 
Erzeugt die Ausgabe: \glqq Freddie Herbie Ron Max\grqq{}
Hier wird zuerst ein Array mit den Namen \glqq Freddie\grqq{}, \glqq Herbie\grqq{}, \glqq Ron\grqq{}, \glqq Max\grqq{} und \glqq Ringo\grqq{} erstellt. Dann wird die Continuation erstellt und in der globalen Variable \$cc gespeichert. Anschließend wird die Continuation ausgeführt, bis die Zeichenfolge \glqq Max\grqq{} enthalten ist, und der Code kehrt zu dem Zeitpunkt zurück, an dem die Continuation erstellt wurde.
Continuations können in bestimmten Anwendungsfällen nützlich sein, um ein Programm dynamisch zu steuern und an verschiedenen Stellen des Codes zurückzukehren. Allerdings kann dies auch die Lesbarkeit des Codes stark beeinträchtigen.
\section*{ Blöcke:}
Blöcke in Ruby sind eine Syntax, die es ermöglicht, anonyme Funktionen zu erstellen und diese an Methoden weiterzugeben. Sie können entweder durch das do..end-Statement oder die geschweiften Klammern \{\} definiert werden. Die Verwendung von Blöcken führt zu kompaktem und wiederverwendbarem Code, da sie es ermöglichen, Funktionen in einem späteren Verlauf erneut zu verwenden. Das Schlüsselwort yield wird verwendet, um den Code innerhalb des Blocks aufzurufen und auszuführen.\footnote{https://www.rubyguides.com/2016/02/ruby-procs-and-lambdas/Aufgerufen am 07.01.2024}
Ein Beispiel illustriert dies:
\begin{lstlisting}
def print_hello 
yield
end

print_hello { puts "Hello" }
\end{lstlisting}
In diesem Beispiel wird die Methode print\_hello definiert, die einen Block akzeptiert und das yield-Statement verwendet, um den darin enthaltenen Code auszuführen. Wenn die Methode mit einem Block aufgerufen wird, wird der darin enthaltene Code ausgeführt und "Hello" wird auf der Konsole ausgegeben.
Was Blöcke besonders macht gegenüber anonymen Funktionen, ist, dass man Closures hat. Das bedeutet, dass man auf Variablen aus dem umgebenden Scope zugreifen kann, auch nachdem der Block beendet wurde.
Außerdem ermöglicht das Keyword yield, den Code eines Blockes an einer beliebigen Stelle des Codes auszuführen.
\section*{Ruby Mixins:}
In Ruby gibt es im Gegensatz zu anderen objektorientierten Sprachen nur einfache Vererbung.
Dagegen gibt es Module, die in Klassen eingebunden werden können. Dies ermöglicht es, Funktionen in einer Klasse zu teilen, ohne eine direkte Vererbungsbeziehung einzugehen.
Module in Ruby sind Sammlungen von Methoden, Konstanten und anderen Ruby-Elementen. Diese Module können dann in Klassen eingebunden werden, um deren Funktionalität zu erweitern.
Ein einfaches Beispiel zeigt den Einsatz von Mixins:
\begin{lstlisting}
module Debug 
  def whoAmI?  
    "#{self.type.name} (\##{self.id}): #{self.to_s}"  
  end  
end 

class Phonograph 
include Debug  
  # ...  
end  

class EightTrack  
  include Debug 
 # ...  
end 

ph = Phonograph.new("West End Blues")  
et = EightTrack.new("Surrealistic Pillow") 

ph.whoAmI? 
# "Phonograph (#537766170): West End Blues" 

et.whoAmI? 
# "EightTrack (#537765860): Surrealistic Pillow"
\end{lstlisting}
Beispiel aus der Ruby-Dokumentation\footnote{https://ruby-doc.com/docs/ProgrammingRuby/html/tut\_modules.htmlAufgerufen am 07.01.2024}\\ 
Hier wird das Debug-Modul eingefügt, und sowohl Phonograph als auch Eighttrack haben nun Zugang zur Instanzmethode whoAmI.
Mixins fördern die Wiederverwendbarkeit von Code, da Module unabhängig voneinander entwickelt und in verschiedenen Klassen wiederverwendet werden können. Dies erleichtert die Strukturierung des Codes und reduziert die Redundanz. Außerdem können mit Mixins Probleme die mit Mehrfachvererbungen auftreten können vermieden werden.Ein Beispiel für ein Problem, das durch Mehrfachvererbung entstehen kann, ist das sogenannte Diamantproblem. Dabei erbt eine Klasse von zwei anderen Klassen, die wiederum von derselben übergeordneten Klasse erben. Dies kann zu Konflikten führen, da die erbende Klasse möglicherweise widersprüchliche oder unerwartete Verhaltensweisen von ihren Elternklassen erbt. Da keine direkte Vererbung bei Mixins stattfindet, können in Ruby solche Konflikte vermieden werden.
Mit Mixins können Entwickler Funktionen hinzufügen, ohne komplexe Vererbungshierarchien zu erstellen. Dadurch ist der Code flexibel und leicht zu pflegen. Ruby Mixins sind ein leistungsstarkes Feature, das die Wiederverwendbarkeit und Flexibilität von Code in der Programmiersprache Ruby fördert. Sie bieten eine gute Möglichkeit, Funktionen zu teilen, ohne die klare Struktur und Lesbarkeit des Codes zu beeinträchtigen.
\section*{Entwicklung von Ruby:}
Als Open-Source-Programmiersprache ermöglicht Ruby Entwicklern aus der ganzen Welt, am Code mitzuwirken und Verbesserungen vorzuschlagen. Der Quellcode von Ruby wird auf GitHub gehostet.\footnote{https://www.ruby-lang.org/de/community/ruby-core/Aufgerufen am 07.01.2024}
Um aktiv an der Entwicklung von Ruby teilzunehmen, können Interessierte sich der Mailingliste von Ruby anschließen. Auf dieser Mailingliste werden wichtige Informationen, Diskussionen und Entscheidungen im Zusammenhang mit der Weiterentwicklung der Sprache geteilt. Es ist zu beachten, dass viele der Diskussionen und Beiträge auf Japanisch verfasst sein können, da viele der ursprünglichen Ruby-Entwickler aus Japan stammen.
Die Veröffentlichung von Ruby-Updates erfolgt in der Regel jährlich um die Weihnachtszeit.
\section*{Dokumentation von Ruby:}

Die Ruby-Dokumentation ist die beste Ressource für Entwickler, um die Sprache effektiv zu lernen, zu verstehen und zu nutzen.\footnote{https://www.ruby-lang.org/de/documentation/Aufgerufen am 07.01.2024}
Die offizielle Website von Ruby bietet eine umfassende Dokumentation sowohl für Anfänger als auch für erfahrene Entwickler.
Rubys offizielle Website bietet verschiedene Arten von Dokumentation.
Dazu gehört zunächst eine eigene Ruby-Dokumentation, die direkt auf der Website verfügbar ist.
Diese Dokumente beschreiben umfassend die Syntax, Funktionen und Best Practices der Sprache.
Sie dienen als zentrale Anlaufstelle für Entwickler, die genaue Informationen über Ruby suchen.
Darüber hinaus trägt eine aktive Ruby-Community zur Dokumentation bei.
Die intelligente Suche bzw.
das intelligente Abrufen von Informationen wird durch die Integration von Suchtools in Ihre Website erleichtert.
Benutzernotizen ermöglichen es Mitgliedern der Ruby-Community außerdem, Erfahrungen, Tipps und zusätzliche Informationen auszutauschen.
Neben grundlegenden Informationen bietet diese Website auch detaillierte Anleitungen, Tutorials und Referenzmaterialien, die es Entwicklern ermöglichen, Ruby von Grund auf zu erlernen oder ihr Wissen zu erweitern.
Außerdem gibt es Werkzeuge wie RDoc, die den Sourcecode analysieren und automatisch Dokumentationsseiten generieren können. Diese enthalten Informationen zu Klassen, Modulen, Methoden und Kommentaren.

\section*{IDEs:}
Es gibt eine Reihe integrierter Entwicklungsumgebungen (IDEs), die die Ruby-Programmierung unterstützen.
Dazu gehören Visual Studio Code, Sublime Text, AWS Cloud9 und RubyMine, das von JetBrains speziell für Ruby entwickelt wurde.
Es ist jedoch erwähnenswert, dass RubyMine kostenpflichtig ist, während einige der anderen genannten IDEs kostenlos verfügbar sind.

\section*{Garbage Collector}
Der Garbage Collector (GC) in Ruby automatisiert die Speicherbereinigung, indem er nicht mehr benötigte Objekte identifiziert und freigibt. Ruby verwendet den Mark-and-Sweep-Algorithmus, um erreichbare Objekte zu markieren und nicht mehr benötigte Objekte zu bereinigen. Der GC durchläuft dabei Phasen der Markierung und Bereinigung, um Speicherlecks zu verhindern und die Gesamtleistung zu verbessern. Zusätzlich wurde in Ruby 2.1 der Generations-Garbage-Collection-Ansatz eingeführt, der den Speicher in verschiedene Generationen unterteilt und eine effiziente Verwaltung ermöglicht.
\section*{RubyGems:}
Die Paketverwaltung von RubyGems  spielt eine zentrale Rolle in der Ruby-Entwicklung, indem sie ein standardisiertes Format für die Bereitstellung von Bibliotheken und Programmen namens „Gems“ bereitstellt.\footnote{https://rubygems.org/?locale=deAufgerufen am 07.01.2024}
 Dieses System erleichtert  die Installation und Aktualisierung von Gems sowie deren Verteilung innerhalb der Ruby-Community.
 RubyGems wird als  standardisiertes Format bereitgestellt, das Entwicklern eine konsistente Möglichkeit ihrer Gems oder Programme bereitzustellen sowie einen Server um diese zu hosten.
 Dies stellt sicher, dass Gems konsistent formatiert sind und ermöglicht einen reibungslosen Austausch zwischen verschiedenen Projekten und Entwicklern.
  Gems werden mit dem Gem-Befehlszeilentool installiert, das der Ruby-Community eine benutzerfreundliche Schnittstelle  bietet.
 Entwickler können  die erforderlichen Bibliotheken herunterladen, installieren und in ihre Projekte integrieren, indem sie einfach ein paar Befehle ausführen.
 Dies vereinfacht die Handhabung von Abhängigkeiten und beschleunigt den Entwicklungsprozess erheblich.
 RubyGems ist seit Ruby Version 1.9  in der Standard-Ruby-Distribution enthalten.

\footnote{https://web.archive.org/web/20220117234323/https://svn.ruby-lang.org/repos/ruby/tags/v1\_9\_1\_0/NEWSAufgerufen am 07.01.2024}
Bis 2010 existierte noch gems.rubyforge.org, welches jedoch ab diesem Zeitpunkt zu RubyGems wechselte.\footnote{https://en.wikipedia.org/wiki/RubyGems\#cite\_note-:0-5Aufgerufen am 07.01.2024}
\section*{Ruby Interpeter:}


Ruby ist eine interpretierte Programmiersprache.
Das heißt, der Code wird vor der Ausführung nicht in Maschinencode übersetzt.
Stattdessen erfolgt die Codeausführung zur Laufzeit über einen Interpreter, der als Schnittstelle zwischen dem geschriebenen Code und dem Prozessor fungiert.
Diese Interpretation erfolgt Zeile für Zeile.
Wenn also ein Fehler auftritt, stoppt der Interpreter sofort in der betroffenen Zeile, was den Debugging-Prozess erleichtert.
Die erste Generation von Ruby-Interpretern war MRI (Matz's Ruby Interpreter), der bis Version 1.
9 verwendet wurde.
MRI wurde vom Ruby-Erfinder Yukihiro Matsumoto in C geschrieben und war aufgrund des Global Interpreter Lock, auch bekannt als GIL, nur begrenzt für die parallele Programmierung geeignet.
GIL verhindert die gleichzeitige Ausführung mehrerer Threads und schränkt die volle Auslastung von Multicore-Prozessoren ein.
In Version 1.
9 wurde MRI durch YARV (Yet Another Ruby VM) ersetzt.
YARV führte einen Just-in-Time-Compiler (JIT) ein, der die Geschwindigkeit der Codeausführung erheblich verbesserte.
Dennoch bleibt die GIL erhalten, was bedeutet, dass eine echte parallele Ausführung von Ruby-Threads nicht möglich ist.

\section*{Andere Ruby Implementierungen:}

Ruby hat nicht nur eine Implementierung, sondern viele Implementierungen.
Die Standardimplementierung wird auch als MRI („Matz's Ruby Interpreter“) oder Cruby bezeichnet, da sie in C geschrieben ist.
Es gibt jedoch auch alternative Implementierungen, die andere Features und Funktionen bieten.
Ein Beispiel ist JRuby, welches in Java geschreiben wurde.\footnote{https://www.ruby-lang.org/en/about/Aufgerufen am 07.01.2024}
Dies ermöglicht die Nutzung von Java Virtual Machine spezifischen Funktionen wie dem JIT-Compiler, Garbage Collector, gleichzeitigen Threads und vielen anderen JVM-Funktionen.
Eine weitere Implementierung ist Rubinius, die in Ruby selbst geschrieben wurde und auf LLVM aufbaut.
Ein weiteres Beispiel ist TruffleRuby, die auf der GraalVM läuft. Ihr Hauptziel ist es, Ruby-Code schneller auszuführen und die parallele Ausführung sowie die kontrollierte Ausführung von C-Erweiterungen zu ermöglichen. TruffleRuby ist Open Source und wird von Oracle gesponsert.
Diese unterschiedlichen Implementierungen geben Ruby-Entwicklern Flexibilität bei der Auswahl der Plattform, die ihren Anforderungen am besten entspricht.
Jede Implementierung bietet einzigartige Vorteile und Funktionen und bereichert die Vielfalt der Ruby-Entwicklung.
Es ist wichtig zu beachten, dass die Wahl der Implementierung oft von den spezifischen Anforderungen des Projekts abhängt, sei es in Bezug auf Leistung, Plattformintegration oder andere Faktoren.
Die meisten entscheiden sich allerdings für eine andere Implementierung, vor allem, weil diese schneller ist und Parallelität unterstützt, ohne dabei Features zu verlieren. 

\section*{Ruby Community:}
Ein weiterer bedeutender Bestandteil ist die Ruby-Community. Neben der Unterstützung, die sie bei der Weiterentwicklung der Sprache bietet, und der Hilfe durch die zahlreichen von der Community entwickelten Gems, spielen auch die vielen Ruby-Foren eine wichtige Rolle.
Neben bekannten Kommunikationsplattformen wie Reddit und Stack Overflow gibt es allein auf der Ruby-Homepage eine Vielzahl von Links zu Ruby User Groups. In diesen Gruppen können sich Programmierende mit anderen austauschen und sich in Gruppen organisieren, die sich in der Regel monatlich treffen.\footnote{https://www.ruby-lang.org/de/community/Aufgerufen am 07.01.2024}
Insbesondere die Mailingliste von Ruby ist ein wichtiger Bestandteil für den Austausch über Ruby und die aktive Mitarbeit an der Ruby-Entwicklung. Es gibt vier englischsprachige Mailinglisten und sogar eine auf Deutsch.\footnote{https://www.ruby-lang.org/de/community/conferences/Aufgerufen am 07.01.2024}
Zusätzlich zu den Foren und Mailinglisten finden jährlich Ruby-Konferenzen statt. Die EuRuKo (European Ruby Conference) wird einmal im Jahr von deutschen Ruby-Enthusiasten organisiert. Die RubyConf, die seit 2001 die jährliche englischsprachige Ruby-Konferenz in den USA ist, sowie die RubyKaigi, eine japanisch und englischsprachige Konferenz seit 2006, bieten Plattformen für den Austausch von Wissen und die Vernetzung innerhalb der Ruby-Community.

\section*{Zukunft:} 
Die kontinuierliche Entwicklung und die engagierte Community tragen dazu bei, dass Ruby auch in Zukunft eine bedeutende Rolle spielen wird. Jährlich erscheinende Updates verbessern sowohl die Leistung als auch die Funktionalität von Ruby erheblich.
In den letzten Jahren war vor allem Ruby on Rails der treibende Faktor für die Verwendung von Ruby, aber mit anderen Programmiersprachen, die zu Ruby on Rails aufholen, stellt sich die Frage, ob Ruby in den nächsten Jahren weiter wachsen kann. Es ist jedoch wichtig zu betonen, dass die Zukunft von Ruby nicht allein von der Existenz von Ruby on Rails abhängt.
Die Vielseitigkeit von Ruby als Sprache und die breite Anwendungsmöglichkeit in verschiedenen Kontexten machen Ruby unabhängig vom spezifischen Framework relevant. Die Sprache hat sich in verschiedenen Szenarien als effektiv erwiesen, darunter Webentwicklung und Aufgabenautomatisierung .
Trotzdem stehen Herausforderungen durch zunehmende Konkurrenz von anderen Sprachen und Frameworks bevor. Entwickler haben heutzutage eine Fülle von Optionen, und die Wahl einer bestimmten Sprache hängt oft von den spezifischen Anforderungen des Projekts ab.


\section*{Grep Clone:}
Die Umsetzung des Programms, das als eine Art Grep-Klon fungiert, unterteilt sich in drei wesentliche Abschnitte. Im ersten Abschnitt stehen die Optionen im Fokus. Hier werden die verfügbaren Befehlszeilenoptionen definiert, und es wird festgelegt, welche Informationen an diese Optionen übergeben werden können. Mit der Bibliothek OptionParser wird dabei das Parsen der Befehlszeilenoptionen erleichtert. Mit OptionParser können die gewünschten Optionen sowie deren Funktionen klar über die Befehlszeile spezifiziert werden. Die Optionen werden dabei in einem Hash definiert und dann mit OptionParser konfiguriert. Dabei geben die Optionen an, ob die Treffer davor und danach angegeben werden sollen, ob die Suche case-insensitive sein soll, ob versteckte Dateien und Ordner durchsucht werden sollen und ob die Treffer farblich hervorgehoben werden sollen. Zunächst werden die Optionen mit Hilfe von OptionParser definiert. Jede Option hat eine kurze und eine lange Form sowie eine Beschreibung, wie zum Beispiel -A und --after-context. Anschließend werden die Befehlsargumente mit der parse! Methode analysiert und den entsprechenden Optionen zugeordnet. 
 \begin{lstlisting}[
    basicstyle=\small, %or \small or \footnotesize etc.
]
 opts.on("-A", "--after-context LINES", Integer,
	 "Prints the given number of following lines for each match") do |a|
        options[:after_context] = a
    end
\end{lstlisting}
Der Hauptteil des Programms ist auf die Suche nach Übereinstimmungen ausgerichtet und gibt Ergebnisse entsprechend der angegebenen Optionen zurück. Wenn die Option aktiviert ist, dass die Treffer farblich hervorgehoben werden sollen, erfolgt die Einfärbung durch die Integration der „Colorize“-Bibliothek, die eine einfache Farbgebung der Ausgabe auf der Befehlszeile ermöglicht.
Die Hauptfunktion search\_file liest die Dateien in Chunks ein, um den Speicherverbrauch zu minimieren, und verarbeitet dann jede Zeile einzeln. Wenn eine Zeile das Muster enthält, wird diese zusammen mit vorherigen und folgenden Zeilen ausgegeben, wenn die Optionen für vorherigen und nachfolgenden Kontext aktiviert sind.
Im letzten Abschnitt wird das Programm als Parallelprozess ausgeführt. Hierbei kommt die „Parallel“-Bibliothek zum Einsatz, die die parallele Ausführung von Programmen erleichtert.
\begin{lstlisting}
num_processes = Etc.nprocessors
Parallel.each(files, in_processes: num_processes) do |file|
    search_file(file, pattern, options)
end
 \end{lstlisting}
Wie bereits erwähnt, ist es aufgrund des Global Interpreter Lock (GIL) in Ruby nicht möglich, echtes paralleles Programmieren durchzuführen.
Der Parallel-Gem bietet jedoch eine Lösung, indem er Prozesse anstelle von Threads verwendet. Aufgrund des GIL kann nur ein Thread gleichzeitig einen Ruby-Prozess ausführen, was bedeutet, dass selbst wenn mehrere Threads in einem Ruby-Programm vorhanden sind, nur einer von ihnen zu einem bestimmten Zeitpunkt ausgeführt wird. Der Parallel-Gem umgeht dieses Problem, indem er mehrere Prozesse verwendet, wobei jeder Prozess seinen eigenen GIL hat. Die „etc“-Bibliothek wird verwendet, um Informationen über die verfügbaren Prozessoren zu erhalten und die Anzahl der für die parallele Ausführung verwendeten Prozesse zu bestimmen.
Das Skript durchsucht dann entweder ein einzelnes Verzeichnis rekursiv oder eine einzelne Datei. Es ignoriert dabei versteckte Dateien, solange nicht in den Optionen angegeben wurde, dass nach diesen gesucht werden soll.
Die before\_context- und after\_context-Optionen wurden so implementiert, dass immer, wenn eine neue Zeile eingelesen wird, diese zu einer Warteschlange hinzugefügt wird. Wenn die Größe der Warteschlange größer ist als die der before\_context-Option, wird die älteste Zeile aus der Warteschlange entfernt. Wenn ein Match gefunden wurde, werden dann alle Zeilen in der Warteschlange ausgegeben, wobei diese dann geprüft werden, ob sie selbst ein Match sind. Falls sie ein Match sind, werden sie ebenfalls ausgegeben. Bei after\_context wird, wenn es eine Übereinstimmung gibt, für jede darauffolgende Zeile bis zur Anzahl der after\_context-Optionen ausgegeben. Da die Zeilen nur den aktuellen Chunk berücksichtigen, wurde ein Rescue eingebaut, der für den Fall, dass die Zeilen mal nicht in einem Chunk sind, trotzdem die Zeile ausgeben kann.
Um in Ruby nach regulären Ausdrücken zu suchen, kann man die Methode match benutzen. Diese sucht nach einem bestimmten Pattern, das zuvor angegeben wurde. Da \textbackslash w nicht ganz so wie bei grep sich verhalten hat, wird, falls \textbackslash w im Pattern ist, dies durch p\{L\} und p\{N\} ausgetauscht, welches nach allen Wörtern und Zahlen sucht. Eine neue Regular Expression wird dann erstellt, und nach dieser kann man dann suchen. Falls die Option ignore\_case an ist, wird mit Hilfe der Methode Regexp::IGNORECASE die Groß- und Kleinschreibung nicht beachtet.
\begin{lstlisting}[
    basicstyle=\small, %or \small or \footnotesize etc.
]
 if options[:ignore_case] && pattern.include?('\w') && pattern.include?('\s')
        pattern = pattern.encode('UTF-8').downcase 
        pattern_without_w = pattern.gsub('\w', '[\p{L}\p{N}_a]')
        
        pattern_regex = Regexp.new(pattern_without_w)  
            
    
    end
\end{lstlisting}
Um Binärdateien zu filtern, verwende ich die Methode binary\_file?, die den MIME-Encoding-Typ einer Datei bestimmt. Dafür nutze ich das Unix-Kommando file -b --mime-encoding. Diese Methode wird dann verwendet, um zu überprüfen, ob eine Datei binär ist, und in diesem Fall wird sie übersprungen. 
\begin{lstlisting}
def binary_file?(file)
    `file -b --mime-encoding #{file}`.strip == 'binary'
end
\end{lstlisting}
Die größten Herausforderungen bei der Programmentwicklung waren vor allem Leistungsprobleme und die Einschränkungen des Hauptinterpreters MRI. Die Leistungsprobleme hätten möglicherweise durch den Einsatz alternativer Ruby-Interpreter wie JRuby oder Rubinius verbessert werden können.
Um die Leistungsprobleme zu analysieren und zu verbessern, habe ich unter anderem das Ruby Gem "Benchmark" verwendet. Durch die Anwendung dieses Gems konnte ich feststellen, dass das Programm beispielsweise unnötigerweise auf die Konvertierung jeder Zeile in UTF-8 gewartet hat, selbst wenn dies nicht erforderlich war.
In meiner vorherigen Implementierung für den After- und Before-Context musste der Text jedes Mal erneut durchlaufen werden, um die entsprechenden Kontextzeilen zu finden. Dies führte zu erheblichen Verzögerungen, da der gesamte Text erneut verarbeitet werden musste. In meiner neuen Umsetzung habe ich dieses Problem behoben, indem ich die Kontextzeilen in einer Warteschlange gespeichert habe. Dadurch musste der Text nicht erneut durchlaufen werden, und die Verarbeitung wurde erheblich beschleunigt.
Trotz dieser Herausforderungen war das Programmiererlebnis mit Ruby äußerst angenehm und leicht verständlich. Die klare Syntax und hohe Lesbarkeit der Sprache erleichtern den Entwicklungsprozess erheblich. Ruby ist schnell erlernbar und bietet einen unkomplizierten Einstieg in die Programmierung, insbesondere für Entwickler mit bereits vorhandener Programmiererfahrung.

\section*{Fazit:}
Die Verwendung von Ruby hat sich im Laufe der Jahre als sehr vielfältig und einflussreich erwiesen, und sein Einfluss erstreckt sich auf ein breites Anwendungsspektrum.
Ruby hat sich aufgrund seiner klaren Syntax, Flexibilität und Fähigkeit, elegante und lesbare Code zu erstellen, in vielen Bereichen als eine der großen Programmiersprachen etabliert.
Die wichtigste Anwedung von Ruby ist allerdings die Webentwicklung, insbesondere durch das beliebte Ruby on Rails-Framework.
Rails hat nicht nur die Entwicklerproduktivität erheblich gesteigert, sondern auch den Standard für die Entwicklung von Webanwendungen gesetzt.
Die Fähigkeit von Ruby on Rails, Konventionen statt Konfigurationen festzulegen, hat eine effiziente und konsistente Entwicklung ermöglicht, was bei vielen Projekten zu einer schnelleren Markteinführung führte.
Darüber hinaus hat die Ruby-Community wichtige Beiträge zur Entwicklung von Bibliotheken und Gems geleistet, die die Fähigkeiten von Ruby erweitern und eine Vielzahl von Anwendungsbereichen abdecken.
Von Datenbank-ORMs (objektrelationales Mapping) bis hin zu Test-Frameworks bieten diese Ressourcen Entwicklern die Tools, die sie zum Erstellen robuster, skalierbarer Anwendungen benötigen.
Ein weiterer wichtiger Aspekt ist die Einfachheit und Lesbarkeit von Ruby-Code.
Eine klare und intuitive Syntax erleichtert Anfängern den Einstieg und ermöglicht erfahrenen Entwicklern das effiziente und ausdrucksstarke Schreiben von Code.
Ein weiteres Merkmal ist die Flexibilität von Ruby, die es den Programmierern ermöglicht, verschiedene Paradigmen wie objektorientierte, funktionale und imperative Programmierung ohne Probleme zu integrieren.
Diese Vielseitigkeit hat es Ruby ermöglicht, in einer Vielzahl von Situationen erfolgreich eingesetzt zu werden, von kleinen Skripten bis hin zu großen Unternehmensanwendungen.
Die globale Ruby-Community spielt eine wichtige Rolle bei der Weiterentwicklung der Sprache.
Durch die Zusammenarbeit an Ruby und die Organisation von Konferenzen, Meetups und Foren fördert die Community den Wissensaustausch und bietet Entwicklern auf der ganzen Welt Unterstützung.
Was die Zukunft von Ruby angeht, sind noch weitere spannende Entwicklungen in Sicht.
Die Einführung neuer Funktionen und Verbesserungen durch Ruby Core-Entwickler zeigt ihr anhaltendes Engagement für die Weiterentwicklung der Sprache.
Diese Innovationen, gepaart mit der fortwährenden Unterstützung durch die Community, lassen darauf schließen, dass Ruby auch in den kommenden Jahren eine wichtige Rolle in der Softwareentwicklung spielen wird. Es ist möglich, dass Ruby in Zukunft weniger genutzt wird, da andere Programmiersprachen die Funktionen und Innovationen von Ruby on Rails übernehmen könnten, was wiederum zu einer Abnahme der Verwendung von Ruby führen könnte.


\end{document}

