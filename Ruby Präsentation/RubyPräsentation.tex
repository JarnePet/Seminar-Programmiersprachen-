\documentclass{beamer}
\usetheme{Boadilla}
\title{Ruby}
\author{Jarne Peters}
\date{\today}
\usepackage{graphicx}
\usepackage{tikz}


\begin{document}

\begin{frame}
\titlepage
\end{frame}

\section{Ruby}



\begin{frame}{Ruby}
\begin{tikzpicture}[overlay, remember picture]
  \node[anchor=north east, inner sep=0pt] at (current page.north east) {\includegraphics[width=0.22\textwidth]{header-ruby-logo.png}};
\end{tikzpicture}
  \begin{itemize}

    \item 1995 von Yukihiro Matsumoto entwickelt.
    \item Motivation: Kombination der besten Features aus Perl, Smalltalk, Lisp und Python.
    \item Einzigartig durch Dynamik und Interpretierbarkeit.
    \item Elegante Organisation von Daten und Funktionen in Form von Objekten.
    \item Ruby ist vor allem wegen seiner einfachen Nutzung beliebt.
    \item Yukihiro Matsumoto legte besonders großen Wert darauf, dass Programmieren mit Ruby Spaß macht.
  \end{itemize}
\end{frame}


\begin{frame}{Entwicklung von Ruby}
  \begin{itemize}
    \item Als Open-Source-Programmiersprache ermöglicht Ruby Entwicklern weltweit, am Code mitzuwirken und Verbesserungen vorzuschlagen.
    \item Der Quellcode von Ruby wird auf GitHub gehostet.
    \item Interessierte können sich der Mailingliste von Ruby anschließen, um aktiv an der Entwicklung teilzunehmen.
    \item Die Mailingliste teilt wichtige Informationen, Diskussionen und Entscheidungen zur Weiterentwicklung der Sprache.
    \item Die Veröffentlichung von Ruby-Updates erfolgt in der Regel jährlich um die Weihnachtszeit.
  \end{itemize}
\end{frame}
\begin{frame}{Dokumentation von Ruby}
  \begin{itemize}
    \item Die Ruby-Dokumentation ist die beste Ressource für Entwickler, um die Sprache effektiv zu lernen, zu verstehen und zu nutzen.
    \item Die offizielle Website von Ruby bietet eine umfassende Dokumentation sowohl für Anfänger als auch für erfahrene Entwickler.
    \item Neben grundlegenden Informationen bietet die Website detaillierte Anleitungen, Tutorials und Referenzmaterialien.
  \end{itemize}
\end{frame}


\begin{frame}{RubyGems: Package Management in Ruby}
  \begin{itemize}
    \item RubyGems stellt ein standardisiertes Format für die Bereitstellung von Bibliotheken und Programmen namens Gems bereit.
  
    \item Dies stellt sicher, dass Gems konsistent formatiert sind und ermöglicht einen reibungslosen Austausch zwischen verschiedenen Projekten und Entwicklern.
    \item Gems werden mit dem Gem-Befehlszeilentool installiert
    \item Entwickler können mit diesen erforderliche Bibliotheken herunterladen, installieren und in ihre Projekte integrieren
    \item RubyGems ist seit Ruby Version 1.9 in der Standard-Ruby-Distribution enthalten.
  \end{itemize}
\end{frame}
\begin{frame}{Andere Ruby-Implementierungen}
  \begin{itemize}
    \item Ruby hat nicht nur eine Implementierung, sondern viele.
    \item Die Standardimplementierung wird als MRI (Matz's Ruby Interpreter) oder Cruby bezeichnet
    \item Es gibt alternative Implementierungen mit verschiedenen Features und Funktionen.
    \item JRuby basiert auf der Java Virtual Machine (JVM) 
    \item Rubinius ist in Ruby selbst geschrieben und baut auf LLVM auf
    \item Truffle Ruby ist eine leistungsstarke Implementierung
    \item Jede Implementierung bietet einzigartige Vorteile und Funktionen, die die Vielfalt der Ruby-Entwicklung bereichern.
    
  \end{itemize}
\end{frame}
\begin{frame}{Ruby Interpreter}
  \begin{itemize}
    \item Ruby ist eine interpretierte Programmiersprache.
    \item Erste Generation von Ruby-Interpretern war MRI (Matz's Ruby Interpreter) bis Version 1.9.
    \item MRI wurde von Yukihiro Matsumoto in C geschrieben, war aber aufgrund des Global Interpreter Lock (GIL) nur begrenzt für parallele Programmierung geeignet.
    \item In Version 1.9 wurde MRI durch YARV (Yet Another Ruby VM) ersetzt, der einen JIT-Compiler einführte und die Codeausführung beschleunigte.
    \item GIL bleibt erhalten, wodurch echte parallele Ausführung von Ruby-Threads nicht möglich ist.
   
  \end{itemize}
\end{frame}
\begin{frame}{Blöcke in Ruby}
  \begin{itemize}
    \item Blöcke in Ruby ermöglichen die Erstellung anonymer Funktionen und deren Weitergabe an Methoden.
    \item Sie können durch das `do..end`-Statement oder geschweifte Klammern `\{\}` definiert werden.

    \item Die Verwendung von Blöcken führt zu kompaktem und wiederverwendbarem Code.
    \item Das Schlüsselwort `yield` ruft den Code innerhalb des Blocks auf

  \end{itemize}
\begin{tikzpicture}[overlay, remember picture]
  \node[anchor=south, inner sep=0pt] at (current page.south) {\includegraphics[width=0.6\textwidth]{block.png}};
\end{tikzpicture}

\end{frame}
\begin{frame}{Ruby Mixins}
\section*{Ruby Mixins:}

\begin{itemize}
  \item In Ruby keine Mehrfachvererbung, stattdessen Module als Mixins.
  \item Mixins werden in Klassen eingefügt, um Funktionalitäten zu teilen, ohne direkte Vererbung.
  \item Im Gegensatz zur klassischen Vererbung erlaubt Ruby, mehrere Module in eine Klasse einzufügen.
 
 
  \item Reduzieren Redundanz, ermöglichen flexible Hinzufügung von Funktionen ohne komplexe Vererbungshierarchien.
  \item Leistungsstarkes Feature für Wiederverwendbarkeit und Flexibilität ohne Beeinträchtigung der klaren Struktur.
\end{itemize}
\end{frame}
\begin{frame}{Ruby Mixins}
\section*{Ruby Mixins:}

\begin{tikzpicture}[overlay, remember picture]
  \node[anchor=south, inner sep=0pt] at (current page.south) {\includegraphics[width=1\textwidth]{mixins.png}};
\end{tikzpicture}

\end{frame}

\begin{frame}{Ruby on Rails}

Entwicklung und Einführung:
  \begin{itemize}
    \item Ruby on Rails (Rails) ist ein Web Application Framework, entwickelt in der Programmiersprache Ruby.
    \item Veröffentlicht 2004, schnell an Beliebtheit gewonnen durch innovative Features.
  \end{itemize}


Model-View-Controller (MVC)-Muster:

  \begin{itemize}
    \item Grundlegendes Konzept, strukturiert Anwendungen in Modell, Controller und Ansicht.
    \item Modell repräsentiert Daten und Beziehungen, Controller handhabt Anfragen, Ansicht visualisiert Daten.
  \end{itemize}

  \begin{itemize}
    \item "Don't Repeat Yourself" (DRY): Informationen nur einmal in der Datenbank, fördert Code-Wiederverwendbarkeit.
    \item "Convention over Configuration": Sinnvolle Standardwerte, minimiert Konfiguration, spart Entwicklungszeit.
  \end{itemize}
\end{frame}



\begin{frame}{Ruby on Rails}
  \begin{itemize}
  \item Fokus auf der Freude am Programmieren, Spaß bei der Entwicklung von Anwendungen.
    \item Nach dem "Bigger Smile" Prinzip gestaltet, um Entwicklern ein positives Erlebnis zu bieten.
    \item Plattformen wie GitHub, Shopify und Twitch nutzen Ruby on Rails.
  \end{itemize}
\end{frame}
\begin{frame}{Grep Clone Implementation}
  \begin{itemize}
    \item Umsetzung eines Grep-Klons in drei Hauptabschnitten.
    \item Optionen, Matching und parallele Ausführung.
    \item Verwendung der OptionParser-Bibliothek für die Definition von Befehlszeilenoptionen.
    \item Farbliche Hervorhebung von Treffern durch Integration der Colorize-Bibliothek.
    \item Hauptfunktion \texttt\{search\_file\} verarbeitet Dateien in Chunks, minimiert Speicherverbrauch.
    \item Parallelisierung mit der Parallel-Bibliothek für effiziente Ausführung.

   
  \end{itemize}
\end{frame}
\begin{frame}{Grep Clone Implementation}
  \begin{itemize}
    
    \item Suche nach regulären Ausdrücken mit der \texttt\{match\}-Methode
    \item Leistungsanalyse mit dem Benchmark-Gem und Identifizierung von Verbesserungsmöglichkeiten.
    \item Herausforderungen waren Leistungsproblemen und Einschränkungen des MRI-Interpreters.
  
    \item Klar strukturierte Syntax und hohe Lesbarkeit erleichtern den Entwicklungsprozess.
    
  \end{itemize}
\end{frame}
\begin{frame}{Grep Clone Implementation}
\vspace{-\topsep}

  \begin{tikzpicture}[overlay, remember picture]
  \node[anchor=center, inner sep=0pt] at (current page.center) {\includegraphics[width=\paperwidth, height=\paperheight, keepaspectratio]{optionen.png}};
\end{tikzpicture}

  
\end{frame}
\begin{frame}{Grep Clone Implementation}

   \begin{tikzpicture}[overlay, remember picture]
  \node[anchor=center, inner sep=0pt] at (current page.center) {\includegraphics[width=\paperwidth, height=\paperheight, keepaspectratio]{chunks.png}};
\end{tikzpicture}
\end{frame}
\begin{frame}{Grep Clone Implementation}

   \begin{tikzpicture}[overlay, remember picture]
  \node[anchor=center, inner sep=0pt] at (current page.center) {\includegraphics[width=\paperwidth, height=\paperheight, keepaspectratio]{before.png}};
\end{tikzpicture}
\end{frame}
\begin{frame}{Grep Clone Implementation}

   \begin{tikzpicture}[overlay, remember picture]
  \node[anchor=center, inner sep=0pt] at (current page.center) {\includegraphics[width=\paperwidth, height=\paperheight, keepaspectratio]{parallel.png}};
\end{tikzpicture}
\end{frame}
\begin{frame}{Fazit: Die Vielseitigkeit und Zukunft von Ruby}
  \begin{itemize}
    \item Ruby hat sich als vielseitig und einflussreich in verschiedenen Anwendungsbereichen bewiesen.
 
    \item Besonderer Fokus auf Webentwicklung durch das beliebte Ruby on Rails-Framework.
    \item Einfachheit und Lesbarkeit von Ruby-Code erleichtern den Einstieg und effizientes Programmieren.
    \item Ruby wird voraussichtlich eine bedeutende Rolle in der Softwareentwicklung beibehalten.
    \item Herausforderungen durch zunehmende Verwendung von Ruby on Rails im Vergleich zu anderen Frameworks.
   
  \end{itemize}
\end{frame}
\begin{frame}{Quellen}
  \begin{itemize}
    \item https://www.ruby-lang.org/
    \item https://rubyonrails.org/
    \item https://www.jetbrains.com/de-de/lp/devecosystem-2021/ruby/
    \item https://ruby-doc.com/docs/ProgrammingRuby/html/tut\_modules.html
    \item https://ruby-doc.com/docs/ProgrammingRuby/html/tut\_modules.html
  \item https://rubygems.org
  \end{itemize}
\end{frame}
\end{document}