\documentclass[titlepage]{article}
\usepackage{babel}
\pagestyle{plain}
\pagenumbering{arabic}
\begin{document}
\textbf
{Outline}

\textbf{1. Einleitung:}

In der Einleitung werde ich auf den Erfinder Yukihiro Matsumoto eingehen, seine Beweggründe für die Entwicklung von Ruby erläutern und darlegen, warum diese Sprache entstand.

\textbf{Geschichte von Ruby:}

Hier werde ich die historische Entwicklung von Ruby beleuchten, aufzeigen, wie sich die Programmiersprache im Laufe der Jahre entwickelt hat, und aktuelle Projekte vorstellen, die in Ruby verfasst sind, wie zum Beispiel Ruvy von Shopify.

\textbf{Besonderheiten von Ruby:}

In diesem Abschnitt werde ich auf die Besonderheiten von Ruby eingehen. Dazu zählt unter anderem, dass Ruby eine interpreted Sprache ist, jeder Wert als Objekt behandelt wird und jede Funktion als Methode fungiert. Des Weiteren werde ich auf die zahlreichen Frameworks wie Ruby on Rails eingehen, das für die Webentwicklung verwendet wird, sowie auf die Bibliotheken, die als Gems bekannt sind, von denen es mehr als 100.000 gibt.

\textbf{Implementierung:}

Hier werde ich die Umsetzung des Programms erläutern. Dabei werde ich auf die verwendeten Bibliotheken eingehen und die Herausforderungen während der Programmierung beschreiben, sowie meine Lösungsansätze für diese Schwierigkeiten darlegen.

\textbf{Fazit:}

Abschließend werde ich erneut auf die Vor- und Nachteile von Ruby eingehen und ein Fazit zur Sprache ziehen.
\end{document}